\newsection{Conclusion}

Ce projet avait pour objectif la conception de la logique de contrôle d’un système de déplacement linéaire basé sur un FPGA. L’ensemble du travail a permis de développer une architecture numérique complète capable de gérer le positionnement précis d’un chariot, le contrôle du moteur à courant continu et l’interface utilisateur.

La logique du système repose principalement sur l’utilisation de compteurs itératifs, d’additionneurs à propagation de report et de machines à états finis (FSM). Ces éléments ont permis de gérer la position absolue du chariot, le sens de rotation du moteur ainsi que les différentes phases de déplacement, telles que l’accélération, la vitesse constante et la décélération. L’utilisation d’un signal PWM a assuré un contrôle efficace de la vitesse du moteur tout en respectant les contraintes matérielles imposées par le pont en H.

L’intégration des capteurs de fin de course et de l’encodeur incrémental a permis d’obtenir un retour fiable sur la position du chariot et de garantir un fonctionnement sécurisé du système. De plus, la hiérarchisation du projet en blocs et sous-blocs a facilité la compréhension, la simulation et le débogage de la logique globale.

Les simulations ont joué un rôle essentiel dans la validation du fonctionnement du système et dans l’identification de problèmes tels que les glitches ou les comportements indésirables liés au matériel réel. Les solutions mises en place, notamment l’utilisation de bascules pour la synchronisation des signaux, ont permis d’améliorer la stabilité et la robustesse du circuit.

En conclusion, ce projet démontre qu'une approche structurée et modulaire permet de concevoir un système numérique complexe et fonctionnel. La logique développée répond aux exigences du cahier des charges et constitue une base solide pour le contrôle précis d’un système électromécanique à l’aide d’un FPGA.


\subsection{Proposition d'améliorations}

Le projet fonctionne comme prévu mais il y'à quelque proposition de amélioration qui vaut la pêne de tenir en compte:
\begin{itemize}
    \item Rotation en fonction de l'encodeur
    \item Vitesse minimale en fonction de l'encodeur
    \item 
\end{itemize}

\subsubsection{Rotation en fonction de l'encodeur}

Utiliser l'encodeur pour donner le signal de count/decount au compteur de position peut être une bonne manière d'éviter des erreurs.

Vu que les signaux A et B sont décalées dans l'encodeur un bloc qui analyse si le flanc montant de A arrive avant celui de B peut très simplement donner le bon signal de rotation.

Ce système utiliserait un concept d'auto-feedback où le moteur bouge en fonction de l'encodeur, et l'encodeur donne les bonnes informations au bloc qui bouge le moteur.

\subsubsection{Vitesse minimale en fonction de l'encodeur}

Actuellement la vitesse minimale a été trouvé en "essayant" des valeurs jusqu'à que le moteur bougeait.
En réalité il est possible d'augmenter le signal PWMProportion dans le compteur relatif sans le signal enable qui vient du bloc \texttt{changeDetector} et utiliser le premier mouvement détecté par l'encodeur pour connaître la vitesse minimale.  

\subsection{Commentaire personal}
Globalement en faisant ce projet on a appris beaucoup et on à surtout renforcé nos connaissances en conception numérique.

Sûrement on aurait pu développer un système plus simple qui remplissait toutes les critères de fonctionnement indiquées dans le cahier des charges, mais 
le défi de créer quelque chose de plus avancée nous à bien intéressé.
L'idée du mode manuel et sauvegarde des position était déjà là depuis le début du projet et ça nous à permis de commencer la création en faisant un système déjà flexible 
et prêt pour l'implémentation des nouvelles fonctions.

Voir le projet marcher était satisfaisant.

Au niveau du rapport, avoir eu fait un système plus complexe nous à rendu le travail plus long et difficile, sourtout vu que l'écriture du rapport nous à pris au totale au moins 
25 heures chacun.
