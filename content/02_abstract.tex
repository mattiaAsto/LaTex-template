
\section*{\vAbstractName}
\noindent

Ce projet détaille la conception en VHDL de la logique de contrôle pour le système projet Cursor. 

L'architecture repose sur l'implémentation de compteurs et d'additionneurs permettant de gérer précisément le positionnement et la génération des signaux PWM. 

Le système coordonne également une interface LCD et des machines à états (FSM) pour assurer le pilotage des moteurs et le retour visuel. 

Ce rapport documente la hiérarchie numérique, la synthèse des composants, ainsi que les erreurs courantes rencontrées lors du 
développement, offrant ainsi une analyse complète des défis techniques liés au fonctionnement global du dispositif.

\vspace{2cm}
\section*{\vContactsName}
\noindent
\textbf{Mattia Astori} \\
E-Mail: \href{mailto:mattia.astori@hes-so.ch}{mattia.astori@hes-so.ch} \\
Tel.: \href{tel:+41797908164}{+41 79 790 81 64}
\vspace{.5cm}

\textbf{Alex Crestin} \\
E-Mail: \href{mailto:alex.crestin@hes-so.ch}{alex.crestin@hes-so.ch} \\
Tel.: \href{tel:+41794660960}{+41 79 466 09 60}
\vspace{.5cm}

\textbf{HEI Sion} \\
Résponsable du cours: Silvan Zahno\\
E-Mail: \href{mailto:silvan.zahno@hes-so.ch}{silvan.zahno@hes-so.ch} \\

