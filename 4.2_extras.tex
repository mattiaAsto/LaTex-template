\newsection{Fonctions supplémentaires}

Les fonction supplémentaires ajoutés au circuit sont les suivantes:

\begin{itemize}
    \item Déplacement manuel du chariot
    \item Sauvegarde de nouvelles positions
    \item LEDs en fonction de la puissance du moteur
    \item Écriture de l'état de la machine sur l'écran \acrshort{LCD}
\end{itemize}

\subsection{LCD}

La figure \ref{fig:06_asciiFSM} montre l'ensemble des blocs traitant l'écran lcd.

\begin{figure}[H]
    \centering
    \includegraphics[width=0.7\linewidth]{pictures/06/06_asciiFSM.png}
    \caption{Blocs dédié à l'ecran}
    \label{fig:06_asciiFSM}
\end{figure}

Le bloc \texttt{lcdController} (bloc de droite) était déjà fourni. Son rôle principal est de gérer la couche physique de la communication. Il reçoit le code \acrshort{ASCII} (bus \texttt{character}) et le découpe bit par bit pour le traitement.

Le bloc \texttt{lcdDisplay} était déjà créé mais a dû être rempli.

\subsubsection{Entrées}
\begin{itemize}
    \item \texttt{motorOn}: signal à 1 quand le moteur tourne
    \item \texttt{lcd\_display}: un bus (4 bits) provenant de \hyperref[subsub:Bloc ButtonControlFSM]{\texttt{buttonControlFSM}}. Il contient les informations pour sélectionner quel message doit être affiché à l'écran \acrshort{LCD}
    \item \texttt{busy}: un signal qui est à 0 quand le bloc \texttt{lcdController} peut recevoir le prochain \texttt{character} à traiter
    \item \texttt{Clock} et \texttt{restart}: horloge et réinitialisation de base
\end{itemize}

\subsubsection{Sorties}
\begin{itemize}
    \item \texttt{Character}: bus (8 bits) qui transmet la valeur numérique du caractère à afficher
    \item \texttt{send}: le bloc \texttt{lcdController} lira l'entrée \acrshort{ASCII} que lorsqu'il verra une impulsion sur le signal \texttt{send}
\end{itemize}

Le bloc \texttt{lcdDisplay} est une grande machine d'état. Son but est de transmettre les messages à afficher sur l'écran \acrshort{LCD}. La figure \ref{fig:06_FSM_ball_content} illustre les actions à effectuer dans une bulle d'état.

\begin{figure}[H]
    \centering
    \includegraphics[width=0.6\linewidth]{pictures/06/06_FSM_ball_content.png}
    \caption{Boule d'état, efface l'affichage}
    \label{fig:06_FSM_ball_content}
\end{figure}

Toutes les bulles d'états ont la même "Action". Seule la valeur entre les parenthèses de \texttt{pos}, en jaune en haut, change. 5 valeurs ont été utilisées dans ce projet:
\begin{itemize}
    \item \texttt{stx}: va à la ligne 0 et le caractère 0
    \item \texttt{can}: efface tout l'affichage
    \item \texttt{lf}: retour à la ligne
    \item \texttt{cr}: va au début de la ligne
    \item \texttt{"x"}: x représente le symbole \acrshort{ASCII} à afficher. Il faut la mettre entre parenthèses.
\end{itemize}

La figure \ref{fig:06_FSM_linetext} est un exemple d'une ligne d'affichage.

\begin{figure}[H]
    \centering
    \includegraphics[width=1\linewidth]{pictures/06/06_FSM_linetext.png}
    \caption{Ligne de texte dans la \acrshort{FSM} de l'écran}
    \label{fig:06_FSM_linetext}
\end{figure}

Dès que la FSM reçoit la valeur \texttt{lcd\_display} = \texttt{"1100"}, l'affichage est effacé (\texttt{can2}). Quand \texttt{busy} = \texttt{"0"}, la prochaine bulle peut être effectuée. La position d'écriture (\texttt{home2}) est déplacée à la ligne 0 et le caractère 0. Chaque caractère d'une phrase a besoin de sa propre bulle. Donc pour écrire \texttt{"save POS"}, il faut 8 bulles.

La documentation explique l'intégralité du domaine avec les autres valeurs disponibles.

La figure \ref{fig:06_FSM_highlighted} regroupe tous les messages à transmettre.

\begin{figure}[H]
    \centering
    \includegraphics[width=1\linewidth]{pictures/06/06_FSM_highlighted.png}
    \caption{Bloc LcdDisplay, color coded}
    \label{fig:06_FSM_highlighted}
\end{figure}

Il y a 6 grandes parties:
\begin{itemize}
    \item \textbf{Cercle noir}: le pivot de tous les messages. C'est un état où la FSM attend de recevoir un code valide du bus \texttt{lcd\_display}.
    \item \textbf{Partie violette}: s'occupe du mode automatique. Dès l'alimentation du système, la FSM commence dans le cercle rouge et affiche le message mode AUTO. Il fait aussi deux retours à la ligne et se place au début de la ligne. Au cercle vert, il repasse au cercle noir et attend qu'un des boutons soit sélectionné. Quand un des boutons est sélectionné, la FSM refait la partie expliquée ci-dessus (cercle rouge $\to$ cercle vert), puis écrit soit ``go to RST'', ``go to POS1'' ou ``go to POS2''. Il s'arrête et attend dans les états encerclés en bleu et orange. Dès qu'il reçoit le signal que le chariot est bien arrivé à sa position finale, la FSM passe dans le prochain état, efface la deuxième ligne et réécrit soit ``RESTART'', ``POSITION1'' ou ``POSITION2''.
    \item \textbf{Partie verte}: s'occupe du mode manuel. Il affiche ``mode MANU'' sur la première ligne quand le bouton 4 est poussé.
    \item \textbf{Partie bleue}: s'occupe de la sauvegarde d'une nouvelle position. Il affiche ``save POS'' quand le bouton 4 a été maintenu pendant plus de 3 secondes.
    \item \textbf{Partie jaune}: s'occupe d'afficher le décompte quand le bouton 4 est maintenu. Il affiche chaque seconde maintenu le message ``hold for'' avec soit ``3'', ``2'' ou ``1'' à la fin du message.
    \item \textbf{Partie rouge}: s'occupe d'afficher la deuxième ligne du mode manuel. Il affiche ``Move left'' quand le bouton 2 est pressé, ``Move right'' quand le bouton 3 est pressé et ``At Limit~!'' si le chariot atteint la limite du parcours. ``At Limit~!'' peut aussi être affiché à partir des pivots (cercles orange de la partie violette) si le chariot se trouve devant les capteurs pendant qu'il se déplace vers la position 1 ou 2.
\end{itemize}

\subsection{LEDs}

Pour représenter la puissance du moteur sur les 8 leds à coté de l'écran \acrshort{LCD}, une \acrshort{FSM} est utilisé.

\begin{figure}[H]
    \centering
    \includegraphics[width=0.7\linewidth]{pictures/06/06_LED_FSM_out.png}
    \caption{Bloc \texttt{ledPowerFSM}}
    \label{fig:06_LED_FSM_out}
\end{figure}

Les inputs du bloc \texttt{ledPowerFSM} sont: \texttt{clock}, \texttt{reset}, \texttt{motorOn} et \texttt{PWMProportion\_OUT} (figure \ref{fig:06_LED_FSM_out}).

La figure \ref{fig:06_LED_FSM} montre la succession des boules dans la \acrshort{FSM}, chaque boule allume un led, la machine démarre dans \texttt{idle}.

\begin{figure}[H]
    \centering
    \includegraphics[width=0.7\linewidth]{pictures/06/06_LED_FSM.png}
    \caption{Machine d'état gestion des LEDs}
    \label{fig:06_LED_FSM}
\end{figure}

Une fois que le moteur est allumé la machine entre dans la boule qui allume le premier led. Pour passer à l'état suivant la puissance (\texttt{PWMProportion\_OUT}) doit dépasser la valeur de 91 (sur 255). Le reste des boules suivent la même logique, une fois que la puissance passe un certain niveau un nouveau led s'allume.
Si la puissance descend la logique fait en sorte que la machine rentre dans un état précédent et un led s'éteint.

Le calcul pour trouver les différentes puissances est le suivant:

\begin{equation}
    step = \frac{P_{max}-P_{min}}{8} = \frac{255-64}{8} \approx 23.88 \rightarrow 23
\end{equation}

L'approximation est vers le bas pour que la dernière puissance ne soit pas supérieur à la puissance maximale.

Pour allumer les leds, un bus à 8 bits est utilisé, le premier bit représente le premier led.

\subsection{Bouton 4}

Le bouton 4 fait deux fonctions:
\begin{itemize}
    \item Choix entre mode manuel et automatique
    \item Sauvegarde de la position
\end{itemize}

Pour distinguer les deux fonctions un countdown est implémenté, si le bouton est allumé pendant plus que 3.5\,s (détails dans \ref{subsub:Countdown counter}), le système rentre en mode sauvegarde, sinon le bouton change entre automatique et manuel\footnote{Pour éviter des boucles infinies, pour aller depuis manuel à automatique le bouton reset doit aussi être activé}.

Une fois que le système est en mode sauvegarde, la nouvelle position peut être sauvegardé soit dans le registre 1 en poussant le bouton go1 soit dans le registre 2 en poussant le bouton go2.
\begin{figure}[H]
    \centering
    \includegraphics[width=1\linewidth]{pictures/06/06_main_extraFSM.png}
    \caption{Extrait de la partie de FSM dédie aux fonctions auxiliaires}
    \label{fig:06_main_extraFSM}
\end{figure}

L'image \ref{fig:06_main_extraFSM} montre la partie dédiée aux extras de la \acrshort{FSM} du \texttt{mainController}, les couleurs signifiant:
\begin{itemize}
    \item Vert: déplacement manuel 
    \item Rouge: partie du countdown, avec l'affichage du countdown sur l'écran
    \item Violet: retour en mode auto
    \item Bleu: sauvegarde dans position 1 ou position 2
    \item Jaune: sécurité, au cas où le chariot arrive aux limites
\end{itemize}

Pour plus d'info sur la structure de la machine d'état voir l'annexe \ref{pdf:button_control_fsm_extras}.

\begin{figure}[H]
    \centering
    \includegraphics[width=0.5\linewidth]{pictures/06/06_motor_extraFSM.png}
    \caption{Partie de la FSM du \texttt{motorControl} dédiée aux extras}
    \label{fig:06_motor_extraFSM}
\end{figure}

La figure \ref{fig:06_motor_extraFSM} montre la partie dédié au mode manuel du \texttt{motorControl}.
Le but de cette boule est de bouger le moteur soit à droite soit à gauche avec une puissance moyenne pour une meilleure précision.

\subsubsection{Countdown counter}
