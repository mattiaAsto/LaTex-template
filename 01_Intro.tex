\newsection{Introduction}
Le projet "curseur" est le projet semestriel pour les étudiants de Systèmes Industrielles dans le cours de conception numérique.

Le projet s'est développé en plusieurs phases, la réalisation de la logique du bloc principal\footnote{"Top level logic"},
la création des sous blocs, le simulation indépendante de chaque sous blocs, la simulation du projet dans son ensemble, l’amélioration et l’implémentation d'autres
fonctions et enfin l'écriture du rapport qui à été fait en parallel au reste.



\subsection{Objectif}
Le but de ce projet est de créer la logique utilisé pour programmer un \acrshort{FPGA}, un circuit logique intégré programmable 
sur le terrain\cite{FPGA_refrence_report}. Le \acrshort{FPGA} doit contrôler le déplacement d'un "chariot" sur une vise sans fin en respectant des positions, des vitesse et des capteurs analogiques.

Dans la creation de la logique un système avec plusieurs niveaux doit être utilisé, c.a.d des créer des blocs qui eux mêmes contiennent des sous blocs, etc\dots



\subsection{Outils}
Le software utilisé pour la programmation est \acrshort{VHDL} Designer\textsuperscript{\textregistered}, 
en couple avec ModelSim\textsuperscript{\textregistered} pour
la simulation.

One-note a aussi été utilisé pour créer et partager les dessins des blocs.


