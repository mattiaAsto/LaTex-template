\newsection{Composants}
Cette partie de rapport sert seulement à donner des informations sur les composantes qui sont fondamentales pour la compréhension de la continuation 
du rapport, pour toute info précise sur les composantes voir le document de labo \cite{cursor}

Le circuit se compose de 3 platines différentes comme montré dans la figure \ref{fig:03_cursor_image_highlighted}.

\begin{figure}[ht]
    \centering
    \includegraphics[width=1\linewidth]{pictures/03/cursor_image_phone_highlighted.png}
    \caption{Les composants du curseur}
    \label{fig:03_cursor_image_highlighted}
\end{figure}

Les parties principales sont:
\begin{enumerate}
    \item Assemblage avec chariot et \acrshort{PCB} (Bleu)
    \item Carte de développement \acrshort{FPGA} (Rouge)
    \item Carte de control avec 8 LEDS e 4 boutons (Vert)
\end{enumerate}

Encadrées en azure sont le moteur, le chariot et l'encodeur et en jaune sont toutes les connections de transmission des signaux et de l'alimentation des platines.

\subsection{Moteur et control du moteur (pont en H)}

Le chariot est propulsé par un moteur à courant continu 12\,V, piloté par un driver en pont en H (L6207) qui permet de gérer le 
sens de rotation et la vitesse. Cette vitesse est modulée via un signal \acrshort{PWM} dont la fréquence ne doit pas excéder 
100\,kHz pour respecter les limites du composant(\ref{sub:Dynamique du déplacement}).
Le moteur entraîne une vis M12 x 1,75, transformant chaque rotation complète en un déplacement linéaire précis de 1,75\,mm.


\subsubsection{Explication fréquence 100kHz}
Selon un document de Portscap \cite{PWMFrequencyReference}, les paramètres a regarder pour choisir la fréquence de la dent de scie sont les suivantes:

Une fréquence majeure de 20\,kHz est conseillée pour éviter des bruits audibles.

Plus la fréquence est haute, plus le moteur réagit bien au signal \acrshort{PWM}, la période de la dent de scie est tellement courte que le signale d'alimentation est presque continu.

La fréquence de 100\,kHz est la limite maximum pour obtenir un bon fonctionnement, à plus de 100\,kHz le "switching" des \glspl{mosfet} du pont H n'est pas garanti et peut donc causer des problèmes.

100kHz est donc le bon compromis entre un fonctionnement fluide du moteur et la sécurité du circuit.

Dans le cas de ce projet, le pont H utilisé est un L6207 de \href{https://www.st.com/content/st_com/en.html}{STMicreoelectronics\textsuperscript{\textregistered}}, 
selon le \href{https://www.st.com/resource/en/datasheet/l6207.pdf}{datasheet} du pont, la fréquence maximale de travail est d'exactement 100kHz, voir extrait dans figure \ref{fig:03_hbridge_dsheet}.

\begin{figure}[ht]
    \centering
    \includegraphics[width=0.5\linewidth]{pictures/03/03_hbridge_dsheet.png}
    \caption{Extrait du datasheet du pont h}
    \label{fig:03_hbridge_dsheet}
\end{figure}


\subsection{Encoder}

L'angle de rotation de la vis est mesuré par un codeur incrémental (AEDB-9140-A12) \cite{cursor}. 
Il génère 500 impulsions par tour (CPR) sur deux canaux, Channel A et Channel B. 
En analysant le déphasage entre ces deux signaux, le système peut déterminer avec précision le sens de rotation et la position exacte du chariot. 
Un troisième canal, Channel I (Index), fournit une impulsion de référence par tour complet. 
Cependant le Channel I n’est pas utilisé.

\subsubsection{Fréquence de clock}

La fréquence d’horloge du système (clock) s’élève à 66 MHz lors de l’utilisation de la carte FPGA-EBS 2 \cite{cursor}.  
Cette fréquence est imposée par l’oscillateur local qui génère le signal de référence nécessaire au cadencement de la puce Xilinx Spartan.




